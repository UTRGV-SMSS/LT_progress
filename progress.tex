\documentclass{article}
\usepackage{geometry}
\geometry{top=0.65in,bottom=0.50in,left=0.75in,right=0.75in}
\usepackage{calc} % for widthof

% \usepackage[activate={true,nocompatibility},final,tracking=true,kerning=true,spacing=true,factor=1100,stretch=10,shrink=10]{microtype}
% \microtypecontext{spacing=nonfrench}

\usepackage[T1]{fontenc}
\usepackage[sc]{mathpazo}
\usepackage[semibold]{raleway}

\usepackage{tikz}

\usepackage{colortbl} % for colored tables
\usepackage{hyperref}
\hypersetup{
    colorlinks=true,
    linkcolor=black,
    filecolor=magenta,
    urlcolor=black,
}


% \IfFileExists{embedall.sty}{%
%   \usepackage[source]{embedall}
%   \IfFileExists{handouts.tex}{\embedfile{handouts.tex}}{}
% }{}




%%%%%%%%%%%%%%%%%%%%%%%%%%%%%%%%%%%%%%%%%%%%%%%%%%%%%%%%%%%%%%%%%%%
% Set indent to zero, but save value just in case
%%%%%%%%%%%%%%%%%%%%%%%%%%%%%%%%%%%%%%%%%%%%%%%%%%%%%%%%%%%%%%%%%%%
\setlength{\parskip}{2pt plus 7pt minus 2pt}
\newlength\tindent
\setlength{\tindent}{\parindent}
\setlength{\parindent}{0pt}
\renewcommand{\indent}{\hspace*{\tindent}}


%%%%%%%%%%%%%%%%%%%%%%%%%%%%%%%%%%%%%%%%%%%%%%%%%%%%%%%%%%%%%%%%%%%%%%%%%%%
% PAGE HEADERS
%%%%%%%%%%%%%%%%%%%%%%%%%%%%%%%%%%%%%%%%%%%%%%%%%%%%%%%%%%%%%%%%%%%%%%%%%%%


\newcommand{\thedate}{\relax}
\renewcommand{\date}[1]{\gdef\thedate{#1}}


\usepackage{ifthen}
\usepackage{fancyhdr} % Required for header and footer configuration

\RequirePackage{lastpage}

\fancypagestyle{toprunning}{%
%\rfoot{\ifthenelse{\value{page}=\pageref{LastPage}}{}{\sffamily{}Turn the Page.}}
\rhead{\sffamily{}Page \thepage~of~\pageref{LastPage}}
%\lhead{\ifthenelse{\value{page}=1}{}{\sffamily{}\handouttitle}}
\lhead{{\sffamily{}\handouttitle}}
\cfoot{}
}

\fancypagestyle{firstpage}{%
\rhead{\sffamily{}Page \thepage~of~\pageref{LastPage}}
\lhead{\ifthenelse{\value{page}=1}{}{\sffamily{}\handouttitle}}
\cfoot{}
}


\pagestyle{fancy}


\fancyhf{}




\renewcommand{\headrulewidth}{0pt}
\renewcommand{\footrulewidth}{0pt} % Removes the rule in the footer

\addtolength{\headheight}{2.5pt} % Increase the spacing around the header slightly

%%%%%%%%%%%%%%%%%%%%%%%%%%%%%%%%%%%%%%%%%%%%%%%%%%%%%
%%%%%%%%%%%%%%%%%%%%%%%%%%%%%%%%%%%%%%%%%%%%%%%%%%%%%

%%%%%%%%%%%%%%%%%%%%%%%%%%%%%%%%%%%%%%%%%%%%%%%%%%%%%
% Define Title Environment
%%%%%%%%%%%%%%%%%%%%%%%%%%%%%%%%%%%%%%%%%%%%%%%%%%%%%

\newcommand{\handouttitle}{\relax}
\let\originaltitle\title
\renewcommand{\title}[1]{\originaltitle{#1}
\gdef\handouttitle{#1}
}


%%%%%%%%%%%%%%%%%%%%%%%%%%%%%%%%%%%%%%%%%%%%%%%%%%%%%


\pagestyle{toprunning}



\usepackage{tcolorbox}
\usepackage{xargs}
\usepackage{xparse}
\usepackage{ifthen}
\usepackage{probsoln}

%%%%%%%%%%%%%%New Commands%%%%%%%%%%%%%%%%%%%%%%%%%%%


% check and x-mark
\usepackage{pifont}% http://ctan.org/pkg/pifont


\newcommand{\defaultcriteria}{%
    \item All calculus work is done correctly. Sufficient work shown. Use exact answers, not decimal approximations.
    \item Uses calculus to arrive at the answer, not graphical or numerical estimates or guesses.
     \item One or two small arithmetic mistakes may be accepted if it does not significantly alter the problem's intent.
}

\newcommandx{\defineLT}[5][2=\relax,5=\defaultcriteria]{%
  \expandafter\newcommand\csname#1\endcsname[1]{%
\begin{tabular}{@{}m{0.7\textwidth}m{0.3\textwidth}@{}}
#3: #4  & \cxI \, \bxI 
\end{tabular}

}}





%%%%%%%%%%%%%%%%%%%%%%%%%%%%%%%%%%%%%%%%%%%%%%%%%%%%%%%%%%%%%%%%%
%%%%%%%%%%%%%%%%%%%%%%%%%%%%%%%%%%%%%%%%%%%%%%%%%%%%%%%%%%%%%%%%%
\defineLT{LI}[*]{L1}{I can evaluate a limit graphically or numerically including one-sided and infinite limits.}[
    \item All calculus work is done correctly. \item Use exact answers, not decimal approximations.
]

\defineLT{LII}[*]{L2}{I can evaluate a limit analytically (using algebra), including one-sided and infinite limits.}

\defineLT{LIII}{L3}{I can recognize points at which a function is (and is not) continuous and can use continuity to evaluate limits.}

\defineLT{LIV}{L4}{I can describe the behavior of a function on an interval using the Intermediate Value Theorem.}

\defineLT{LV}{L5}{I can identify limits in indeterminate form and apply L'Hopital's rule correctly. }

\defineLT{DMI}{DM1}{I can understand the limit definition of the derivative and calculate the derivative at a point.}


\defineLT{DMII}[*]{DM2}{I can calculate and interpret instantaneous rates of
  change at a point using graphs and tables, and I can understand the difference
  between the instantaneous rate of change and the average rate of change.}

\defineLT{DMIII}{DM3}{I can interpret the average and instantaneous rate of
change using secant and tangent lines.}

\defineLT{DMIV}[*]{DM4}{I can sketch the derivative function from the graph of a
given function (specifically $f'$ from $f$).}


\defineLT{DMV}[*]{DM5}{I can find the tangent line to a function at a given
point.}


\defineLT{DMVI}{DM6}{I can recognize points at which a function is (and is not)
  differentiable, and can use the definition or interpretation of the derivative
to support my thinking.}


\defineLT{DMVII}{DM7}{%
I can use tangent lines to approximate function values, when appropriate.
}

\defineLT{DSI}[*]{DS1}{%
I can compute derivatives correctly using the constant, constant multiple, the
power rules, and sum and difference rules of power functions.
}

\defineLT{DSII}[*]{DS2}{%
I can use the Product and Quotient Rules to compute derivatives of simple algebraic, trigonometric, exponential, and logarithmic functions in combination.
}


\defineLT{DSIII}[*]{DS3}{%
I can use the Chain Rule to compute derivatives of simple algebraic, trigonometric,exponential, and logarithmic functions in combination.
}


\defineLT{DSIV}{DS4}{%
I can use a combination of the Product, Quotient, and Chain Rules to compute derivatives of simple algebraic, trigonometric, exponential, and logarithmic functions in combination.% (Sections 3.5, 3.6, 3.9)
}


\defineLT{DSV}{DS5}{%
I can compute derivatives correctly using implicit differentiation.
}

\defineLT{DSVI}{DS6}{%
I can compute derivatives correctly using logarithmic differentiation.
}

\defineLT{DAI}{DA1}{I can correctly interpret the meaning of a derivative in
context (e.g. velocity, acceleration).}


\defineLT{DAII}[*]{DA2}{I can use calculus to find local and absolute extrema of functions.}


\defineLT{DAIII}{DA3}{I can interpret the meaning of the Mean Value Theorem.}


\defineLT{DAIV}[*]{DA4}{I can explain the relationship between a function and its
  first and second derivatives (concavity, increasing/decreasing, points of
inflection).}


\defineLT{DAV}{DA5}{%
I can solve related rates problems.
}





\title{Math 2413: Learning Targets Progress Tracker}
\date{Fall 2021}


\begin{document}
\maketitle


\newcommand{\bxI}{\tikz \draw[thick] (0,0) rectangle ++(1.0,1.0);}
\newcommand{\cxI}{\tikz \draw[thick] (0cm,0cm) circle(0.50cm);}
\newcommand{\bxII}{\tikz \draw[thick] (0,0) rectangle ++(1.0,1.0) rectangle ++(1.0, -1.0);}
\newcommand{\bxIII}{\tikz \draw[thick] (0,0) rectangle ++(1.0,1.0) rectangle ++(1.0, -1.0) rectangle ++ (1.0, 1.0);}
\newcommand{\bxIV}{\tikz \draw[thick] (0,0) rectangle ++(1.0,1.0) rectangle ++(1.0, -1.0) rectangle ++ (1.0, 1.0) rectangle ++(1.0, -1.0);}

\vfill

\LI


\begin{tabular}{@{}m{0.7\textwidth}m{0.3\textwidth}@{}}
Group L: I can calculate, use, and explain the idea of Limits. & Check Check Test \\
\midrule
L.1: I can evaluate a limit graphically or numerically including one-sided and infinite limits. & \bxII \\
L.2: I can evaluate a limit analytically (using algebra), including one-sided and infinite limits. & \bxIII \\
L.3: I can recognize points at which a function is (and is not) continuous, and can use continuity to evaluate limits. &\bxII \\
\end{tabular}

\vfill

\begin{tabular}{@{}m{0.7\textwidth}m{0.3\textwidth}@{}}
Group DM: I can calculate, use, and explain the idea of Limits. & Check Check Test  \\
\midrule
DM.1: I know the limit definition of the derivative and can explain the purpose of each symbol in
          the definition. & \bxIII \\
DM.2: I can calculate and interpret derivatives and estimates of derivatives using difference quotients
          (including average and instantaneous velocity) & \bxII\\
DM.3: I can explain the connection between average and instantaneous rates of change, and
          can interpret them using secant and tangent lines and limits. & \bxII\\
DM.4: I can find the tangent line to a function at a given point. & \bxIII\\
DM.5: I can recognize points at which a function is (and is not) differentiable, and can use the
          definition or interpretation of the derivative to support my thinking. & \bxIII\\
DM.6: I can use tangent lines to approximate function values. & \bxII\\
\end{tabular}

\vfill

\begin{tabular}{@{}m{0.7\textwidth}m{0.3\textwidth}@{}}
Group DS: I can use multiple strategies to calculate derivatives efficiently.  & Check Check Test  \\
\midrule
DS.1:  I can compute derivatives correctly using the constant, constant multiples, the power rules, and sum and difference rules of power functions.  & \bxIII \\
DS.2: I can compute derivatives of algebraic, trigonometric, exponential, logarithmic, and inverse trigonometric functions correctly using multiple rules in combination. & \bxIII \\
DS.3: I can compute derivatives correctly using the product, quotient, and chain rules. & \bxIII \\
DS.4: I can compute derivatives correctly using implicit differentiation. & \bxIII
\end{tabular}

\vfill

\newpage`



\begin{tabular}{@{}m{0.7\textwidth}m{0.3\textwidth}@{}}
Group DA: I can use derivatives to understand and solve applications. & Check Check Test \\
\midrule
DA.1: I can correctly interpret the meaning of a derivative in context.  & \bxIII \\
DA.2: I can use calculus to find relative and absolute extrema and points of inflection of functions.  & \bxIII \\
DA.3: I can recognize and explain the relationships among the behaviors of $f$, $f^\prime$, and $f^{\prime\prime}$, including slopes, rates of change, and concavity.  & \bxIII \\
DA.4: I can use the information provided by $f$, $f^\prime$, and/or $f^{\prime\prime}$,  to identify and draw accurate graphs of the other functions. & \bxIII \\
DA.5: I can solve related rates problems completely and correctly.  & \bxIII \\
DA.6: I can solve optimization problems completely and correctly.  & \bxIII
\end{tabular}

\vfill

\begin{tabular}{@{}m{0.7\textwidth}m{0.3\textwidth}@{}}
Group FTC: I can calculate and explain the meaning of integrals. & Check Check Test \\
\midrule
FTC.1: I can correctly anti-differentiate basic functions and identify antiderivatives. & \bxIII \\
FTC.2: I can evaluate definite integrals exactly by using graphs and geometry.& \bxIII \\
FTC 3:  I can estimate the values of definite integrals numerically using the left-hand sum or the right-hand sum.   & \bxIII \\
FTC.4: I can evaluate definite integrals exactly by using the Fundamental Theorem of Calculus (FTC) part 2 with an antiderivative. & \bxIII \\
FTC.5: I can interpret the physical meaning of a definite integral in terms of net area, net change, displacement, or distance, and state its units.  & \bxIII \\
FTC.6: I can explain and work with the FTC, part 1 to evaluate derivatives of integrals, including using functions defined by integrals. & \bxIII \\
FTC 7:  I can evaluate integrals using the Substitution Rule.  & \bxIII \\
FTC 8: I can use antidifferentiation to solve initial-value problems. & \bxIV
\end{tabular}

\vfill



\end{document}
